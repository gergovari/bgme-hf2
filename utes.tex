\subsection{Méretezés ütésszerű nyomatékokra}

Az alábbi számolás alapján erre is megfelel a tengelykapcsoló.

\begin{align}
	&J_\text{k} 
	= \siunit{\kapcsJ}{\kilo\gram\meter^2}\\
	&J_\text{A}
	= J_\text{Mot} + J_\text{k}
	= \MtwoJmot + \kapcsJ
	= \siunit{\JA}{\kilo\gram\meter^2} \\
	&J_\text{L}
	= J_\text{ex.} + J_\text{k}
	= \j + \kapcsJ
	= \siunit{\JL}{\kilo\gram\meter^2} \\
	&M_\text{A} 
	= \frac{J_\text{L}}{J_\text{A} + J_\text{L}}
	= \frac{\JL}{\JA + \JL}
	= \siunit{\MA}{-}
\end{align}
\begin{align}
	&T_\text{AS} 
	= 2\cdot T_\text{KN}
	= 2 \cdot \TKN
	= \siunit{\TAS}{\newton\meter}\\
	&S_\text{A} 
	= \siunit{\kapcsSA}{-} \\
	&T_\text{S} 
	= T_\text{AS} \cdot M_\text{A} \cdot S_\text{A}
	= \TAS \cdot \MA \cdot \kapcsSA
	= \siunit{\TS}{\newton\meter}
\end{align}
\begin{align}
	S_\text{Z} 
	&= \siunit{\kapcsSZ}{-}\\
	T_\text{N}
	&= 0\\
	{T_\text{K}}_\text{max}
	&> T_\text{S} \cdot S_\text{Z} \cdot S_\text{t} 
	+ T_\text{N} \cdot S_\text{t} \\
	& \TKMAX > \TS \cdot \kapcsSZ \cdot \St \\
	\siunit{\TKMAX}{\newton\meter} 
	&> \siunit{\TK}{\newton\meter}
\end{align}

\begin{center}
        \begin{tabular}{l}
		$J_\text{k}$: tengelykapcsoló 
		tehetetlenségi nyomatéka\protect\footnote{A \href{https://www.ktr.com/catalog/index.php?catalog=DriveTechnology}{KTR katalógusban} található táblázat alapján. (60. oldal, "ROTEX Complete coupling types")}
		\siunit{}{\kilo\gram\meter^2} \\
		$J_\text{A}$: motor oldal 
		tehetetlenségi nyomatéka \siunit{}{\kilo\gram\meter^2} \\
		$J_\text{L}$: hajtott oldal 
		tehetetlenségi nyomatéka \siunit{}{\kilo\gram\meter^2} \\
		$M_\text{A}$: tehetetlenségi nyomaték együttható \siunit{}{-} \\\\
		$T_\text{AS}$: indítási nyomaték \siunit{}{\newton\meter} \\
		$T_\text{S}$: kezdeti nyomaték \siunit{}{\newton\meter} \\
		$S_\text{A}$: terhelési tényező\protect\footnote{A \href{https://www.ktr.com/catalog/index.php?catalog=DriveTechnology}{KTR katalógusban} található táblázat alapján. (15. oldal, "Shock factor")} \siunit{}{-} \\\\
		${T_\text{K}}_\text{max}$: kezdeti nyomatéklöket a rendszeren \\a terhelési nyomaték ráadása nélkül\protect\footnote{A \href{https://www.ktr.com/catalog/index.php?catalog=DriveTechnology}{KTR katalógusban} találhatóak alapján lett az értéke választva. (33. oldal)} \siunit{}{\newton\meter} \\
		$S_\text{Z}$: indítási tényező\protect\footnote{A \href{https://www.ktr.com/catalog/index.php?catalog=DriveTechnology}{KTR katalógusban} található táblázat alapján. (15. oldal, "Start-up factor")} \siunit{}{-} \\
		$T_\text{N}$: állandó nyomaték \siunit{}{\newton\meter} \\
        \end{tabular}
\end{center}
