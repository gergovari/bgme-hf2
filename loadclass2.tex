\subsubsection{Terhelési osztály}

\begin{equation}
	m_\text{af} 
	= \frac{J_\text{ex.red.}}{J_\text{Mot.}} 
	= \frac{J_\text{ex.}}{J_\text{Mot}} 
	\cdot \left(\frac{1}{I_\text{ges}}\right)^2
	= \frac{\j}{\MtwoJmot} 
	\cdot \left(\frac{1}{\MtwoIges}\right)^2
	= \siunit{\Mtwomaf}{-}
\end{equation}

\begin{center}
        \begin{tabular}{l}
		$m_\text{af}$: tömeg gyorsulás tényező \siunit{}{-} \\
		$J_\text{ex.red.}$: teljes külső tehetetlenségi nyomaték\\ motortengelyre redukálva \siunit{}{\kilo\gram\m^2} \\
		$J_\text{Mot}$: motor tehetetlenségi nyomatéka\protect\footnote{\href{https://www.nord.com/media/documents/bw/m7000_en_600602_2.pdf}{M7000} szabványból. (C3 oldal)} \siunit{}{\kilo\gram\m^2} \\
		$J_\text{ex.}$: munkagép tehetetlenségi nyomatéka \siunit{}{\kilo\gram\m^2} \\
        \end{tabular}
\end{center}

Ezen tényező szerint a rendszer az "A" terhelési osztályba sorolható.\footnote{A \href{https://www.nord.com/media/documents/bw/g1012_ie1_ie2_ie3_en_2.pdf}{NORDBLOC.1 G1012} szabványban található táblázatból. (A6 oldal, "Load Classification")}
