\subsection{Méretezés átvitt nyomatékra}

Az alábbi számolás alapján erre megfelel a tengelykapcsoló.

\begin{align}
	&T_\text{KN} 
	= \siunit{\TKN}{\newton\meter} \\
	&T_\text{N} 
	= 9550 \cdot \frac{P_\text{Mot}}{n}
	= \siunit{\TN}{\newton\meter} \\
	&S_\text{t} = \siunit{\St}{-} \\
\end{align}
\begin{align}
	T_\text{KN} &> T_\text{N} \cdot S_\text{t} \\
	\TKN &> \TN \cdot \St \\
	\siunit{\TKN}{\newton\meter} &> \siunit{\TN}{\newton\meter}
\end{align}

\begin{center}
        \begin{tabular}{l}
		$T_\text{KN}$: tengelykapcsoló által biztosított nyomaték\protect\footnote{A \href{https://www.ktr.com/catalog/index.php?catalog=DriveTechnology}{KTR katalógusban} találhatóak alapján lett az értéke választva. (33. oldal)} \siunit{}{\newton\meter} \\
		$T_\text{N}$: tengelykapcsolóra átadódó nyomaték \siunit{}{\newton\meter} \\
		$S_\text{t}$: hőmérsékleti tényező\protect\footnote{A hajtómű csak $\siunit{40}{\celsius}$-ig használható (\href{https://www.nord.com/media/documents/bw/g1012_ie1_ie2_ie3_en_2.pdf}{NORDBLOC.1 G1012} katalógus, A5 oldal), a \href{https://www.ktr.com/catalog/index.php?catalog=DriveTechnology}{KTR katalógusban} található táblázat alapján. (40. oldal, "ROTEX Steel (SI)", Rated torque)} \siunit{}{-} \\
        \end{tabular}
\end{center}
