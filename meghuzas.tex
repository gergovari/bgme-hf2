\section{Csavar előfeszítése és meghúzási nyomatéka}

\subsection[Csavar szabvány]{Csavar szabvány\protect\footnote{\href{https://www.k-mechanic.hu/kmchnc17/wp-content/uploads/2021/04/Csavarok.pdf}{ISO 4014} szabvány alapján kapott értékek.}}
\begin{align*}
	& p = \siunit{\csavarp}{\mm} \\
	& {d_3}_\text{cs} = \siunit{\csavardthree}{\mm} \\
	& {d_2}_\text{cs} = \siunit{\csavardtwo}{\mm} \\
	& d_w = \siunit{\csavardw}{\mm} \\
	& b = \siunit{\csavarb}{\mm} \\
	& l = \siunit{\csavarl}{\mm} \\
	& \beta = \siunit{\csavarbeta}{\degree}
\end{align*}

\begin{align}
	&\mu_{\substack{\text{min}\\\text{max}}}
	= \substack{
		\siunit{\csavarumin}{-} \\
		\siunit{\csavarumax}{-} \\
	} \\
	&\mu 
	= \frac{\mu_\text{min} + \mu_\text{max}}{2} 
	= \frac{\siunit{\csavarumin}{-} + \siunit{\csavarumax}{-}}{2} 
	= \siunit{\csavaru}{-}
\end{align}

\begin{center}
	\begin{tabular}{l}
		$p$: menet emelkedése \siunit{}{\mm} \\
		${d_3}_\text{cs}$: orsó magátmérője \siunit{}{\mm} \\
		${d_2}_\text{cs}$: csavar középátmérője \siunit{}{\mm} \\
		$\beta$: menetprofil szöge \siunit{}{\degree} \\
		$\mu_{\substack{\text{min}\\\text{max}}}/\mu$: súrlódási tényező\footnote{\href{https://web.archive.org/web/20190713023654/http://www.sze.hu/~szalai/szabvanyok/Anyagok.pdf}{MSZ EN 24014} szabvány alapján kapott értékek.}\textsuperscript{,}\footnote{A súrlódási tényező átlagolható elég nagy biztonsági tényező mellett.} \siunit{}{-} \\
	\end{tabular}
\end{center}

\newpage
\subsection[Meghúzási nyomaték]{Meghúzási nyomaték\protect\footnote{A feladathoz mellékelt segédletből származó számítások. (9-10. oldal)}}

$\alpha$ menetemelkedési szög számítható eddigi adatainkból. A látszólagos súrlódási félkúpszög ($\rho^{'}$) pedig az ismert súrlódási tényezőkből. A csavar meghúzásához szükséges nyomaték ($M_\text{meghúzási}$) a csavar mentén ($M_\text{csavar}$) -és az anya homlokfelületén ($M_\text{anya}$) ébredő súrlódás összege.

\begin{align}
	&\alpha 
	= \arctan{\frac{p}{{d_2}_\text{cs} \pi}} 
	= \arctan{\frac{\csavarP}{\csavardtwo \cdot \pi}}
	= \siunit{\csavaralpha}{\degree} \\
	&\mu^{'}
	= \frac{\mu}{\cos{\frac{\beta}{2}}} 
	= \frac{\csavaru}{\cos{\frac{\siunit{\csavarbeta}{\degree}}{2}}} 
	= \siunit{\nutick}{\radian} \\
	&\rho^{'} 
	= \arctan{\mu^{'}}
	= \arctan{\siunit{\nutick}{\radian}}
	= \siunit{\csavarpt}{\degree} \\
	&d_a 
	= \frac{d_w + d_\text{cs}}{2} 
	= \frac{\csavardw + \karimaM}{2} 
	= \siunit{\csavarda}{\mm}
\end{align}

\begin{equation}
	\begin{split}
	{M_\text{csavar}} 
	&= F_v \frac{{d_2}_\text{cs}}{2} \tan{\left(\alpha + {\rho^{'}}\right)}  \\
	&= \csavarfv \cdot \frac{\csavardtwo}{2} \tan{\left(\siunit{\csavaralpha}{\degree} + \siunit{\csavarpt}{\degree}\right)} \\
	&= \siunit{\csavarMcs}{\newton\mm}
	\end{split} 
\end{equation}
\begin{equation}
	{M_\text{anya}} 
	= F_v \frac{d_a}{2} {\mu^{'}}  
	= \csavarfv \cdot \frac{\csavarda}{2} \cdot {\nutick}  
	= \siunit{\csavarMa}{\newton\mm} \\
\end{equation}

\begin{equation}
	\begin{split}
		{M_\text{meghúzási}} 
		&= {M_\text{csavar}} + {M_\text{anya}} \\
		&= \csavarMcs + \csavarMa 
		= \siunit{\csavarmeg}{\newton\mm}
	\end{split}
\end{equation}

\begin{center}
	\begin{tabular}{l}
		$\alpha$: menetemelkedés szöge \siunit{}{\degree} \\
		$\mu_{\substack{\text{min}\\\text{max}}} / \mu$: súrlódási tényező \siunit{}{-} \\
		$\beta$: menetprofil szöge \siunit{}{\degree} \\
		$d_a$: anya felvekvő felület középátmérője \siunit{}{\mm} \\
		$d_\text{cs}$: csavar szabványos mérete \siunit{}{\mm} \\
		${d_2}_\text{cs}$: menet középátmérője \siunit{}{\mm} \\
		${M_\text{csavar}}$: menet súrlódási nyomatéka \siunit{}{\newton\mm} \\
		$F_v$: csavar terhelése \siunit{}{\newton} \\
		$\rho^{'}$: látszólagos súrlódási félkúpszög \siunit{}{\degree} \\
		${M_\text{anya}}$: csavaranya felülete alatti súrlódási nyomaték \siunit{}{\newton\mm} \\
		${M_\text{meghúzási}}$: meghúzási nyomaték \siunit{}{\newton\mm} \\
	\end{tabular}
\end{center}
